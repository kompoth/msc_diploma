\documentclass{beamer}
\usepackage[utf8x]{inputenc}
\usepackage[T2A]{fontenc}
\usepackage[english,russian]{babel}

\usetheme{mybeamer}

\title{Влияние точности входных параметров моделей ядерных реакций на предсказанные выходы $r$-процесса}
\subtitle{Магистерская диссертация}
\author{Негребецкий В.В.}
\institute{МГУ им. М.В. Ломоносова, физический факультет,\\
кафедра общей ядерной физики}
\date{}

\begin{document}
  \frame[noframenumbering] {
    \titlepage
  }
  
  \frame{
    \frametitle{Нуклеосинтез тяжелых элементов}
    Общее про $r$-процесс:
    \begin{itemize}
      \item Что такое
      \item Почему важен
      \item Условия протекания
    \end{itemize}
  }

  \frame{
    \frametitle{Моделирование нуклеосинтеза}
    Главное о модели:
    \begin{itemize}
      \item Численная задача
      \item Вычислительные трудности
      \item Неопределенности входных параметров
    \end{itemize}
  }

  \frame{
    \frametitle{Астрофизические скорости реакций}
    О скоростях:
    \begin{itemize}
      \item Что это такое
      \item Как влияют
      \item Как вычислять
    \end{itemize}
  }


\usebeamertemplate{endpage}
\end{document}
