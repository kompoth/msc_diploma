\subsection{Моделирование процессов нуклеосинтеза}
\label{sec:nucleosynthesis}
Математически расчет нуклеосинтеза состоит в решении системы обыкновенных дифференциальных уравнений (ОДУ) первого порядка, задающих эволюцию концентраций каждого задействованного в процессе изотопа. Такая система уравнений строится для зоны ограниченного объема, в рамках которой звездное вещество считается однородным. Поиск решения существенно затруднен рядом факторов: огромными размерами системы ОДУ, большим разбросом значений скоростей ядерных реакций, неопределенностью входных параметров. Существует множество методов решения подобных систем уравнений, однако почти все они обладают ограниченной применимостью. Расчет ядерных превращений нуклеосинтеза является одной из наиболее сложных задач такого рода.

\subsubsection{Система уравнений нуклеосинтеза}
Эволюция концентрации каждого отдельного изотопа в ядерной астрофизической системе описывается дифференциальным уравнением следующего вида:
\begin{equation}
\displaystyle \frac{d y_i}{d t} = \sum_{k \in K_i} \pm \lambda_k \prod_{l \in L_k} y_l,
\label{eq:num-1iso}
\end{equation}
где $\lambda_k$ --- скорость $k$-й реакции, $y_i$ --- концентрация $i$-го изотопа,  $K_i$ --- множество всех реакций, в которых $i$-й изотоп фигурирует в качестве исходного или продукта, $L_k$ --- множество исходных изотопов $k$-й реакции. Знак $\pm$ перед каждым элементом суммы~\ref{eq:num-1iso} зависит от того, нарабатывается или расходуется $i$-й изотоп в $k$-й реакции. Ясно, что если изотоп расходуется в реакции, то он входит в $L_k$ не менее одного раза.

Скорость реакции $\lambda_k$ является важнейшим параметром расчета, так как именно через нее характеристики ядерных реакций влияют на эволюцию концентраций изотопов. Как говорилось выше, эти характеристики и, следовательно, значения $\lambda_k$ в случае реакций $r$-процесса известны из теоретических ядерных моделей, что вносит в моделирование нуклеосинтеза существенные неопределенности. Исследование этих неопределенностей составляет задачу настоящей работы.

Записанные для каждого изотопа, эти уравнения образуют систему:
\begin{equation}
\displaystyle
\frac{d \bm{y}}{d t} = \bm{f}(\bm{y}), \qquad \bm{y} \bigg\rvert_{t=0} = \bm{y}_0,
\label{eq:num-system}
\end{equation}
где $\bm{y}$, $\bm{f}$ --- векторы значений концентраций изотопов и правых частей уравнений~(\ref{eq:num-1iso}) соответственно, а вектор $\bm{y}_0$ содержит начальные концентрации изотопов. 

Например, для реакции \({^{12}}\text{C} (\alpha,\gamma) {^{16}}\text{O}\), протекающей со скоростью $\lambda$, фрагмент системы уравнений~(\ref{eq:num-system}) записывается следующим образом:
\begin{equation}
\begin{aligned}
  \dot{y}({^4}\text{He}) &= - \lambda y({^4}\text{He})y({^{12}}\text{C}) + ...\\
  \dot{y}({^{12}}\text{C}) &= - \lambda y({^4}\text{He})y({^{12}}\text{C}) + ...\\
  \dot{y}({^{16}}\text{O}) &= + \lambda y({^4}\text{He})y({^{12}}\text{C}) + ...,
\end{aligned}
\end{equation}
где знак $...$ обозначает вклад других реакций в изменение концентраций этих ядер. Неизвестными здесь являются все концентрации изотопов $y$, кроме их начальных значений. В реальной астрофизической системе число изотопов исчисляется тысячами, и каждому соответствует такое уравнение. Таким образом размерность матрицы системы уравнений~(\ref{eq:num-system}) в реалистичной модели нуклеосинтеза превышает сотни тысяч отличных от нуля элементов.

Системы уравнений, подобные~(\ref{eq:num-system}), часто встречаются в науке, например, в задачах химической кинетики. Для их решения создано множество численных методов, но ни один из них не является универсальным. Наиболее широко применяются классические явные численные методы, в частности семейство методов Рунге-Кутты, к которым относится известный метод Эйлера. Эти методы неприменимы к задаче нуклеосинтеза из-за ее сверхжесткости. Жесткими называют системы дифференциальных уравнений, решение которых при помощи явных численных методов дает неконтролируемый рост ошибки, который не может быть устранен путем уменьшения шага интегрирования. В случае задачи нуклеосинтеза высокая жесткость обусловлена обусловлена большим разбросом коэффициентов уравнений, то есть величин скоростей протекания реакций $\lambda$, --- далее мы увидим, что астрофизические скорости ядерных реакций могут отличаться друг от друга на многие порядки. Это приводит к высокой неустойчивости системы уравнений~(\ref{eq:num-system}) при решении ее классическими явными численными методами.

\subsubsection{Неявный метод Эйлера}
Для решения систем уравнений высокой жесткости широко применяют неявные численные методы, которые обладают большей устойчивостью, но также и меньшей эффективностью.

При численном решении системы дифференциальных уравнений вида~(\ref{eq:num-system}) интервал времени симуляции разбивается на шаги интегрирования. Размер шагов может быть задан заранее, но для большей устойчивости размер каждого следующего шага вычисляют динамически, исходя из различных требований: на гладкость решения, на оценку ошибки и др. Вектор решения (в данном случае, концентраций изотопов) на каждом шагу вычисляется с помощью численной схемы, аппроксимирующей операцию дифференцирования в левой части системы~(\ref{eq:num-system}). Классический метод Эйлера предлагает простейшую численную схему вида
\begin{equation}
\displaystyle
\frac{d \bm{y}}{d t}(t) = 
\frac{\bm{y}(t + \Delta t) - \bm{y}(t)}{\Delta t} = \bm{f}(\bm{y}(t))
\end{equation}
Как видно, значение решения $\bm{y}(t + \Delta t)$ на каждом следующем шаге явно выражается из численной схемы и вычисляется тривиально. При всей простоте данный метод не может быть использован в моделировании нуклеосинтеза, так как из-за высокой жесткости задачи для достижения приемлемой точности пришлось бы делать слишком маленькие шаги по времени.

Неявные методы отличаются от явных тем, что из их численной схемы не получается выразить решение на следующем шагу $\bm{y}(t + \Delta t)$ явно. Вместо этого получается уравнение относительно $\bm{y}(t + \Delta t)$, которое приходится решать на каждом шагу интегрирования. В неявной модификации метода Эйлера схема интегрирования выглядит так: 
\begin{equation}
\displaystyle
\frac{d \bm{y}}{d t}(t + \Delta t) = 
\frac{\bm{y}(t + \Delta t) - \bm{y}(t)}{\Delta t} = 
\bm{f}(\bm{y}(t + \Delta t))
\end{equation}
Из этого выражения величина $\bm{y}(t + \Delta t)$ отсюда не может быть явно выражена. Для ее вычисления приходится решать систему линейных уравнений, например, с помощью процедуры Ньютона, что существенно увеличивает трудоемкость интегрирования. 

Ограничиваясь одной итерацией метода Ньютона, можно получить тривиальное уравнение относительно изменения вектора концентраций изотопов $\Delta \bm{y} = \bm{y}(t + \Delta t) - \bm{y}(t)$:
\begin{equation}
\displaystyle
  \left( \frac{I}{\Delta t} - J \right) \Delta \bm{y} = \bm{f}(\bm{y}_n),
\end{equation}
где $I$ --- единичная матрица, $J$ --- якобиан правых частей системы уравнений ($J_k^i = \partial f_i/\partial y_k$), на вычисление которого также уходит большое число операций.

Хотя неявные численные методы являются значительно более трудоемкими, чем явные, они более устойчивы на задачах высокой жесткости и могут применяться для моделирования нуклеосинтеза. В настоящей работе для расчета $r$-процесса мы будем использовать библиотеку SkyNet~\cite{lippuner2017}, которая использует простой неявный метод Эйлера. Существуют и более продвинутые неявные методы, подходящие для симуляции нуклеосинтеза, например, используемый в пакете модулей астрофизического моделирования MESA~\cite{paxton2011} неявный метод более высокого порядка точности. Кроме того, для задач химической кинетики разработан ряд специальных явных численных методов, устойчивых для решения жестких задач, например,~\cite{bulatov2018}. В работе~\cite{guidry2013} подобный метод успешно применен к задаче термоядерного горения.

В данной работе мы решили остановиться на простом и надежном методе, реализованном в SkyNet, и сосредоточиться на исследовании влияния неопределенностей ядерных данных на расчет нуклеосинтеза.

\subsubsection{Уравнение состояния}
\label{sec:eos}
При интегрировании системы уравнений~(\ref{eq:num-system}) вне зависимости от выбранного численного метода необходимо учитывать влияние термодинамических параметров на процессы нуклеосинтеза. В частности, величины скоростей реакций $\lambda_k$ из уравнения~(\ref{eq:num-1iso}) являются, как будет показано далее, функциями температуры среды и, следовательно, времени. 

При моделировании нуклеосинтеза начальные макроскопические параметры (начальная температура, плотность и энтропия вещества) обычно известны из астрофизических расчетов. Также как минимум для одного из этих параметров должен быть задан закон изменения во времени, зависящий от выбранного сценария. Например, вещество, выброшенное в пространство при столкновении двух нейтронных звезд, можно считать равномерно расширяющимся~\cite{korobkin2012}, поэтому временной профиль плотности в модели $r$-процесса будет выглядеть как $\rho \sim t^{-3}$. Уравнение состояния вещества, в котором протекает нуклеосинтез, позволяет рассчитывать все остальные термодинамические параметры на каждом шаге интегрирования.

В библиотеку SkyNet~\cite{lippuner2017}, с помощью которой мы моделируем процессы нуклеосинтеза, входит уравнение состояния~\cite{timmes1999,timmes2000}, разработанное специально для расчета звездных событий. Эта модель состоит из трех независимых компонент: для фотонов, для электронов с позитронами и для тяжелых ионов. Термодинамические характеристики релятивистского электрон-позитронного газа вычисляются на основе уравнения свободной энергии Гельмгольца с помощью разработанной авторами модели процедуры интерполяции, позволяющей получать точные результаты за разумное время. Уравнение состояния газа тяжелых ионов соответствует невырожденному нерелятивистскому больцмановскому газу.

Отметим, что помимо термодинамических закономерностей при моделировании нуклеосинтеза необходимо учитывать энерговыделение ядерных реакций. Вещество, выброшенное в пространство в результате слияния нейтронных звезд, должно остывать вследствие расширения, однако $\beta$-распады $r$-процесса обеспечивают приток энергии в термодинамическую систему. В SkyNet самонагрев вещества вследствие ядерных реакций учитывается за счет добавки энерговыделения к тепловой энергии системы.

\subsubsection{Статистическое равновесие}
\label{sec:nse}
В том случае, когда сильные астрофизические реакции уравновешены с обратными (например, нейтронный захват уравновешивается с фотодиссоциацией), достигается состояние статистического равновесия, при котором практически не происходит изменения концентраций изотопов в системе. Это происходит, например, при температурах свыше 5~ГК, как отмечается в статье~\cite{lippuner2017}.  

Когда система находится в состоянии статистического баланса, для химических потенциалов каждого $i$-го изотопа и его нуклонов выполняется соотношение
\begin{equation}
  \mu_i = Z_i \mu_p + N_i \mu_n
  \label{eq:nse1}
\end{equation}

Для прямой $\alpha$ и обратной $\alpha'$ реакций условие статистического баланса записывается следующим образом:
\begin{equation}
\lambda_\alpha \prod_{k \in L_\alpha} y_k = \lambda_{\alpha'} \prod_{l \in L_{\alpha'}} y_l,
\label{eq:nse2}
\end{equation}
где $L_\alpha$ --- множество исходных изотопов реакции $\alpha$. Из этого уравнения, зная зависимость концентраций частиц $y_i$ от термодинамических параметров (например, в случае больцмановского газа ионов), можно получить выражение для скорости обратной реакции, которое, как отмечается в~\cite{lippuner2017}, будет справедливо и для неравновесной системы. 

Кроме того, вместе уравнения (\ref{eq:nse1}) и (\ref{eq:nse2}) дают дополнительную связь между концентрациями изотопов и термодинамическими параметрами в равновесной системе. В библиотеке SkyNet эта связь используется для вычисления равновесного состава вещества.
