\section{Подготовка ядерных данных для расчета $r$-процесса} 
\label{sec:reaclib}
Ядерные характеристики, такие как энергия связи, определяют величины скоростей ядерных реакций, непосредственно влияющих на протекание $r$-процесса. Для моделирования $r$-процесса необходимо располагать значениями скоростей нейтронного захвата, $\beta$-распада и других реакций, протекающих при астрофизических условиях. При этом ошибки определения ядерных масс могут приводить к большим неопределенностям в расчете скоростей реакций и, как следствие, сказываться на результатах симуляций механизмов нуклеосинтеза.

Настоящий раздел посвящен описанию проведенных нами расчетов скоростей $(n,\gamma)$, анализу полученных результатов, а также методики формирования из них баз данных астрофизических ядерных реакций, которые можно использовать при моделировании $r$-процесса.

\input tex/pt3_reaclib/talys.tex
\input tex/pt3_reaclib/weakfit.tex
\input tex/pt3_reaclib/ratelib.tex
