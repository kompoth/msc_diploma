\subsection{Моделирование процессов нуклеосинтеза}
Математические расчет нуклеосинтеза состоит в решении системы обыкновенных дифференциальных уравнений (ОДУ) первого порядка, задающих эволюцию концентраций каждого задействованного в процессе изотопа. Поиск решения существенно затруднен рядом факторов: огромными размерами системы ОДУ, большим разбросом значений скоростей ядерных реакций, неопределенностью входных параметров. Существует множество методов решения подобных систем уравнений, однако почти все они обладают ограниченной применимостью. Задача моделирования нуклеосинтеза среди таких задач является одной из наиболее трудоемких.

\subsubsection{Система уравнений нуклеосинтеза}
Эволюция концентрации каждого отдельного изотопа в ядерной астрофизической системе описывается дифференциальным уравнением следующего вида:
\begin{equation}
\displaystyle \frac{d y_i}{d t} = \sum_{k \in K_i} \pm \lambda_k \prod_{l \in L_k} y_l,
\label{eq:num-1iso}
\end{equation}
где $\lambda_k$ --- скорость $k$-й реакции, $y_i$ --- концентрация $i$-го изотопа,  $K_i$ --- множество всех реакций, в которых $i$-й изотоп фигурирует в качестве исходного или продукта, $L_k$ --- множество исходных изотопов $k$-й реакции. Знак $\pm$ перед каждым элементом суммы~\ref{eq:num-1iso} зависит от того, нарабатывается или расходуется $i$-й изотоп в $k$-й реакции. Ясно, что если изотоп расходуется в реакции, то он входит в $L_k$ не менее одного раза.

Скорость реакции $\lambda_k$ является важнейшим параметром расчета, так как именно через нее характеристики ядерных реакций влияют на эволюцию концентраций изотопов. Как говорилось выше, эти характеристики и, следовательно, значения $\lambda_k$ в случае реакций $r$-процесса известны из теоретических ядерных моделей, что вносит в моделирование нуклеосинтеза существенные неопределенности. Исследование этих неопределенностей составляет задачу настоящей работы.

Записанные для каждого изотопа, эти уравнения образуют систему:
\begin{equation}
\displaystyle
\frac{d \pmb{y}}{d t} = \pmb{f}(\pmb{y}), \qquad \pmb{y} \bigg\rvert_{t=0} = \pmb{y}_0,
\label{eq:num-system}
\end{equation}
где $\pmb{y}$, $\pmb{f}$ --- векторы значений концентраций изотопов и правых частей уравнений~(\ref{eq:num-1iso}) соответственно, а вектор $\pmb{y}_0$ содержит начальные концентрации изотопов. 

Для реакции \({^{12}}\text{C} (\alpha,\gamma) {^{16}}\text{O}\), протекающей со скоростью $\lambda$, фрагмент системы уравнений~(\ref{eq:num-system}) записывается следующим образом:
\begin{equation}
\begin{aligned}
  \dot{y}({^4}\text{He}) &= - \lambda y({^4}\text{He})y({^{12}}\text{C}) + ...\\
  \dot{y}({^{12}}\text{C}) &= - \lambda y({^4}\text{He})y({^{12}}\text{C}) + ...\\
  \dot{y}({^{16}}\text{O}) &= + \lambda y({^4}\text{He})y({^{12}}\text{C}) + ...
\end{aligned}
\end{equation}
Неизвестными здесь являются все концентрации изотопов $y$, кроме их начальных значений. В реальной астрофизической системе число изотопов исчисляется тысячами, и каждому соответствует такое уравнение. Таким образом размерность матрицы системы уравнений~(\ref{eq:num-system}) в реалистичной модели нуклеосинтеза превышает сотни тысяч.

Системы уравнений, подобные~(\ref{eq:num-system}), часто встречаются в науке, например, в задачах химической кинетики. Для их решения создано множество численных методов, но ни один из них не является универсальным. Наиболее широко применяются классические явные численные методы, в частности семейство методов Рунге-Кутты, к которым относится известный метод Эйлера. Эти методы неприменимы к задаче нуклеосинтеза из-за ее сверхжесткости. Жесткими называют системы дифференциальных уравнений, решение которых при помощи явных численных методов дает неконтролируемый рост ошибки, который не может быть устранен путем уменьшения шага интегрирования. В случае задачи нуклеосинтеза высокая жесткость обусловлена обусловлена большим разбросом коэффициентов уравнений, то есть величин скоростей протекания реакций $\lambda$. Далее мы увидим, что астрофизические скорости ядерных реакций могут отличаться друг от друга на многие порядки. Это приводит к высокой неустойчивости системы уравнений~(\ref{eq:num-system}) при решении ее классическими явными численными методами.

\subsubsection{Неявный метод Эйлера}
Для решения систем уравнений высокой жесткости широко применяют неявные численные методы, которые обладают большей устойчивостью, но также и большей трудоемкостью.

В настоящей работе для расчета 
