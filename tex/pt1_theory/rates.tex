\subsection{Астрофизические скорости ядерных реакций}

Для моделирования процессов нуклеосинтеза необходимо знать астрофизические скорости всех задействованных ядерных реакций. Скоростью ядерной реакции $\lambda$ называют вероятность ее протекания в единицу времени на единцу концентрации каждой исходной частицы. Величина $\lambda$ напрямую определяется сечением реакции, которые в случае астрофизических процессов в большинстве случаев не могут быть получены экспериментально.



\subsubsection{Интегральная формула скорости реакции}

Скорость ядерной реакции $\lambda$ рассчитывается путем свертки ее сечения с энергетическим распределением взаимодействующих частиц. Энергии нейтронов и ядер в равновесной астрофизической системе имеют распределение Максвелла-Больцмана. Кроме того при астрофизических температурах ядра находятся в возбужденных состояниях, и в условиях термодинамического равновесия заселенность уровней также должна подчиняться статистике Максвелла-Больцмана. Тем самым формула для астрофизической скорости ядерной реакции, являющейся функцией температуры среды $T$, принимает вид
\begin{equation}
\displaystyle
\lambda(T) = \sqrt{\frac{8}{\pi m}} \frac{N_A}{(k T)^{3/2} G(T)} \int_0^\infty \sum_\mu \frac{(2 I^\mu + 1)}{(2 I^0 + 1)} \sigma^\mu(E) E \exp \left( - \frac{E + E_x^\mu}{kT} \right) dE,
\label{eq:rate}
\end{equation}
где $E^\mu_x$ и $I^\mu$ --- энергия и спин возбужденного уровня $\mu$, $m$ --- приведенная масса взаимодействующих частиц, $k$ --- постоянная Больцмана, $N_A$ --- число Авогадро, $G(T)$ --- статистическая сумма
\begin{equation}
    \displaystyle
    G(T) = \sum_\mu \frac{(2 I^\mu + 1)}{(2 I^0 + 1)} \exp \left( - \frac{E_x^\mu}{kT} \right)
\end{equation}

\subsubsection{Связь скоростей обратных реакций}
