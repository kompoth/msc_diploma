\subsection{Статистическая модель ядерных реакций}
Для моделирования процессов нуклеосинтеза необходимо знать астрофизические скорости $\lambda$ всех задействованных ядерных реакций. Скоростью ядерной реакции называют вероятность ее протекания в единицу времени на единцу концентрации каждой исходной частицы. Величина $\lambda$ напрямую определяется сечением реакции, значения которого в случае астрофизических процессов не могут быть получены экспериментально, поэтому их приходится вычислять с помощью теоретических моделей.

При астрофизических температурах ядерные реакции протекают в основном с образованием составного ядра. Вычисление сечений таких реакций может быть выполнено в рамках статистического подхода. Для расчета сечений и астрофизических скоростей нейтронного захвата в настоящей работе используется статистическая модель Хаузера--Фешбаха~\cite{hauser1952}, реализованная в пакете моделирования ядерных реакций TALYS~\cite{koning2019}.

\subsubsection{Интегральная формула скорости реакции}
Скорость ядерной реакции $\lambda$ рассчитывается путем свертки ее сечения с энергетическим распределением взаимодействующих частиц. Энергии нейтронов и ядер в равновесной астрофизической системе имеют распределение Максвелла-Больцмана. Кроме того при астрофизических температурах ядра находятся в возбужденных состояниях, и в условиях термодинамического равновесия заселенность уровней также должна подчиняться статистике Максвелла-Больцмана. Тем самым формула для астрофизической скорости ядерной реакции, являющейся функцией температуры среды $T$, принимает вид
\begin{equation}
\displaystyle
\lambda(T) = \sqrt{\frac{8}{\pi m}} \frac{N_A}{(k T)^{3/2} G(T)} \int_0^\infty \sum_\mu \frac{(2 I^\mu + 1)}{(2 I^0 + 1)} \sigma^\mu(E) E \exp \left( - \frac{E + E_x^\mu}{kT} \right) dE,
\label{eq:rate}
\end{equation}
где $E^\mu_x$ и $I^\mu$ --- энергия и спин возбужденного уровня $\mu$, $m$ --- приведенная масса взаимодействующих частиц, $k$ --- постоянная Больцмана, $N_A$ --- число Авогадро, $G(T)$ --- статистическая сумма
\begin{equation}
    \displaystyle
    G(T) = \sum_\mu \frac{(2 I^\mu + 1)}{(2 I^0 + 1)} \exp \left( - \frac{E_x^\mu}{kT} \right)
\end{equation}

Далее обсудим статистическую модель, используемую для расчета сечений астрофизических ядерных реакций.

\subsubsection{Составное ядро}
Промежуточное составное ядро образуется в тех случаях, когда время пролета частицы через ядро-мишень оказывается меньше времени протекания реакции. В этом случае энергия поглощенной частицы быстро распределется по объему ядра-мишени, и вероятность того, что какой-то нуклон получить достаточную энергию, чтобы преодолеть нуклон-нуклонное взаимодействие и покинуть ядро, оказывается мала. Образуется квазистационарная промежуточная система --- составное ядро, которое существует достаточно долгое время, чтобы его распад никак не зависел от входного канала реакции.

Механизм составного ядра для реакции захвата частицы $a$ ядром $A$ с вылетом из конечного ядра $B$ частицы $b$ можно представить в виде схемы
\begin{equation}
a + A \xrightarrow{\alpha} C^* \xrightarrow{\alpha'} b + B,
\label{eq:reac}
\end{equation}
где $\alpha$ и $\alpha'$ --- входной и выходной каналы реакции. К такому типу реакций относится рассматриваемый в настоящей работе астрофизический радиационный захват нейтрона $(n,\gamma)$. 

Для реакции~(\ref{eq:reac}) могут быть записаны законы сохранения
\begin{equation}
\begin{aligned}
E_a + E_A = E_C = E_b + E_B& \qquad \text{энергии,}\\
I + l + s = J = I' + l' + s'& \qquad \text{углового момента,}\\
\pi_a \pi_A (-1)^l = \Pi = \pi_b \pi_B (-1)^{l'}& \qquad \text{четности,}
\end{aligned}
\end{equation}
где $E_k$ --- энергии частиц, $I$ --- спин состояния ядра-мишени $A$, $l$ и $s$ --- орбитальный момент и спин налетающей частицы, $\pi_k$ --- четности частиц. Величины $J$ и $\Pi$ --- спин и четность составной системы. Штрихами отмечены угловые моменты, относящиеся к выходному каналу.

\subsubsection{Формула Хаузера--Фешбаха}
Согласно гипотезе Бора, образование и распад составного ядра являются независимыми процессами:
\begin{equation}
\displaystyle
\sigma^C_{\alpha \alpha'} = \sigma^C_{\alpha}P_{\alpha'} = 
\sigma^C_{\alpha} \, \frac{\Gamma_{\alpha'}}{\Gamma}
\end{equation}
где $\sigma_{\alpha}$ --- сечение образования составного ядра во входном канале $\alpha$, $P_{\alpha'}$ --- вероятность распада промежуточной системы по каналу $\alpha'$, $\Gamma_{\alpha'}$ и $\Gamma$ --- ширина канала $\alpha'$ и полная ширина распада составного ядра.

Величина $\sigma_{\alpha,\alpha'}$ связывает большое число начальных и конечных состояний системы, причем само это состояние в общем случае складывается из гигантского числа узких резонансов. При вычислениях сечений производится усреднение по всем этим состояниям, причем ширины каждого такого состояния $T_\alpha$ связаны с коэффициентами прохождения $T_\alpha$, которые могут быть рассчитаны из оптической модели.

Из квантовой теории столкновений в общем случае сечение какого-либо канала $\alpha, \alpha'$ реакции может быть записано как
\begin{equation}
\displaystyle
\sigma_{\alpha \alpha'} = \pi \frac{1}{k^2} 
\frac{2J + 1}{(2I + 1)(2s + 1)}
|\delta_{\alpha \alpha'} - S_{\alpha \alpha'}|^2,
\end{equation}
где $S_{\alpha \alpha'}$ --- матричный элемент $S$-матрицы, соответствующий асимптотической амплитуде выходного канала реакции. Среднее значение этой величины напрямую связано с $T_\alpha$:
\begin{equation}
\displaystyle
T_\alpha \equiv 1 - |\bar{S}_{\alpha\alpha}|^2
\end{equation}

В окончательной форме формула Хаузера--Фешбаха для реакции составного ядра может быть записана следующим образом:
\begin{equation}
\begin{gathered}
\displaystyle
\sigma_{\alpha \alpha^\prime} = 
%D^\text{comp}
\pi \frac{1}{k^2}
\sum_{J=\bmod(I + s, 1)}^{l_\text{max} + I + s} \,
\sum_{\Pi=-1}^1 \frac{2J + 1}{(2I + 1)(2s + 1)} \\ \times
\sum_{j=|J-I|}^{J+I} \,
\sum_{l=|j-s}^{j+s} \,
\sum_{j^\prime=|J-I^\prime|}^{J+I^\prime} \,
\sum_{l^\prime=|j-s^\prime|}^{j^\prime+s^\prime} \,
\delta_\pi(\alpha) \delta_\pi(\alpha^\prime) \\ \times
\frac{
  \expval{T_{\alpha l j}^J(E_a)}
  \expval{T_{\alpha^\prime l^\prime j^\prime}^J(E_{a'})}
}{
  \sum_{\alpha'',\, l'',\, j''} \delta_\pi(\alpha'') 
  \expval{T_{\alpha'' l'' j''}^J(E_{a''})}
}
W^J_{\alpha l j \, \alpha' l' j'}
\end{gathered}
\label{eq:talys_cs}
\end{equation}

Множитель $W^J_{\alpha l j \, \alpha' l' j'}$, так называемый множитель флуктуаций ширин, отвечает здесь за поправки усреднения состояний, а множители $\delta_\pi(\alpha)$, $\delta_\pi(\alpha')$ отвечают за сохранение четности:
\begin{equation}
\delta_\pi(\alpha) = \begin{cases}
1, \;\text{если}\; \pi_a \pi_A (-1)^l = \Pi\\
0, \;\text{если}\; \pi_a \pi_A (-1)^l \neq \Pi
\end{cases}
\end{equation}

Таким образом в TALYS вычисление сечений ядерных реакций, протекающих через составное ядро, производится суммированием вкладов всех разрешенных законами сохранения входных и конечных каналов и состояний промежуточного составного ядра.

%В TALYS усреднение величин $T_{\alpha^\prime l^\prime j^\prime}^J(E_{a'})$ делается только в области сильного сгущения уровней. Оно вычисляется путем свертки $T_{\alpha^\prime l^\prime j^\prime}^J(E_{a'})$ с плотностью уровней в узком диапазоне энергий вблизи энергии возбуждения. В области изолированных состояний усреднение не производится. Поправка $W^J_{\alpha l j \, \alpha' l' j'}$ называется множителем флуктуации ширин, с ее помощью учитываются возможные ошибки усреднения состояний. Множитель $D^\text{comp}$ используется в TALYS для учета прямых и предравновесных процессов.
