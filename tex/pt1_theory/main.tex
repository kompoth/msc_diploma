\section{Теоретические модели в расчетах нуклеосинтеза} \label{theormodels}
Моделирование астрофизического $r$-процесса состоит из двух этапов. Сперва рассчитываются характеристики задействованных ядерных реакций, которые не могут быть получены экспериментально. Затем эти данные используются для симуляции эволюции астрофизической ядерной системы вследствие процессов нуклеосинтеза. В настоящем разделе мы пойдем в обратном порядке: сперва объясним, как производится симуляция нуклеосинтеза и как именно в ней используются параметры ядерных реакций, а затем обсудим метод расчета этих параметров.

\input tex/pt1_theory/nucleosynthesis.tex
\input tex/pt1_theory/statmodel.tex

