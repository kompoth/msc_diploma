\section*{Заключение}\addcontentsline{toc}{section}{Заключение}
В рамках настоящей работы нами были рассчитаны скорости реакций нейтронного захвата, задействованных в $r$-процессе, с использованием трех различных моделей ядерных масс: макро-микроскопической FRDM2012~\cite{moller2016}, микроскопической HFB-24~\cite{goriely2013} и модели LMR2021~\cite{vladimirova2022}, основанной на локальных массовых отношениях. С помощью подготовленного нами пакета ratelib на основе этих данных были составлены три библиотеки астрофизических скоростей реакций в формате REACLIB~\cite{reaclib2010}. Полученные базы данных мы использовали для моделирования $r$-процесса в слиянии нейтронных звезд с помощью написанной нами программы, использующей библиотеку симуляции астрофизических ядерных систем SkyNet~\cite{lippuner2015}. Полученные нами результаты говорят высокой чувствительности $r$-процесса к значениям масс нейтроноизбыточных изотопов. Чувствительность эта сказывается не только на выходах $r$-изотопов, но и на самом течении $r$-процесса, потому что вариация массовой модели может значительно влиять на условия статистического баланса между реакциями нейтронного захвата и фотодиссоциации. Кроме того, было обнаружено заметное влияние выбора массовой модели на термодинамику системы, обеспечиваемое в первую очередь энерговыделением $\beta$-распадов, являющихся важной частью цепочек $r$-процесса.

При больших достижениях в определении основных астрофизических сценариев $r$-процесса детальное исследование наблюдаемых распространенностей тяжелых изотопов во Вселенной и получение точных моделей их синтеза все еще являются важными и актуальными задачами современной физики. Как мы видим, связь астрофизических явлений и ядерных реакций нуклеосинтеза очень тесна и сложна, изучение одного невозможно без изучения другого. Многое зависит от развития наших знаний о физике нейтроноизбыточных ядер, задействованных в $r$-процессе. Как показывают наши расчеты, неточность ядерных данных в моделировании нуклеосинтеза может привести к высоким неопределенностям не только распространенностей $r$-изотопов, но и термодинамических параметров, напрямую влияющих на астрофизические расчеты. Все это мотивирует на дальнейшее развитие ядерных моделей, годящихся для предсказания характеристик экзотических изотопов.
