\section{Массы и энергии связи нейтроноизбыточных изотопов} \label{massmodel}
Массы взаимодействующих и результирующих частиц являются важнейшими параметрами расчета сечений ядерной реакции при помощи статистической модели. В частности, энергия связи ядра используется при расчете плотности высоколежащих уровней, которая в модели Ферми-газа задается формулой
\begin{equation} \displaystyle
  \omega(E) = \frac{\sqrt{\pi}}{12} 
  \frac{\exp [ 2\sqrt{a(E-\Delta)} ] }
       {a^{1/4}(E-\Delta)^{5/4}},
\end{equation}
где $a$ и $\Delta$ являются параметрами модели, а $E$ есть энергия возбуждения системы. В случае реакции нейтронного захвата энергия возбуждения $E$ равна сумме кинетической энергии нейтрона и энергии его отделения $S_n$, которая в свою очередь определяется как разница энергий связи $B(N,Z)$ конечного и исходного изотопа:
\begin{equation}
S_n = B(N, Z) - B(N - 1, Z)
\end{equation}

При этом, как видно из работы~\cite{sobiczewski2018}, предсказания масс экзотических изотопов при помощи различных ядерных моделей существенно различаются между собой. Важно исследовать влияние этих неопределенностей на расчеты сечений астрофизических ядерных реакций и на результаты моделирования $r$-процесса. Для этого мы рассмотрим три модели, позволяющие рассчитывать массы ядер и использующие разные подходы.

По методу описания ядерной материи теоретические модели делятся на коллективные, или макроскопические, и микроскопические. Коллективный подход, к которому относится, например, известная формула Вайцзеккера, рассматривает ядро как единое целое, например, как каплю несжимаемой жидкости. В его пользу свидетельствуют деформации ядер, колебательные и вращательные полосы в спектрах. При необходимости описать более тонкие одночастичные эффекты прибегают к микроскопическим моделям, а также коллективным моделям с микроскопическими поправками. Микроскопические модели зачастую основаны на принципе эффективного потенциала, сводящем задачу взаимодействия многих тел к задаче невзаимодействующих тел в общем потенциале.

Для расчета астрофизических скоростей реакций мы используем результаты трех ядерных моделей, реализующих разные подходы к описанию ядерной материи: макро-микроскопической модели FRDM2012~\cite{moller2016}, микроскопической модели HFB-24~\cite{goriely2013} и феноменологического метода локальных массовых соотношений LMR2021~\cite{vladimirova2022}. Настоящий раздел посвящен их обзору и сравнению.

\input tex/pt2_masses/frdm.tex
\input tex/pt2_masses/hfb.tex
\input tex/pt2_masses/lmr.tex
