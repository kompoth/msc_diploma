\subsection{Макро-микроскопическая модель FRDM2012}
Массовая модель FRDM2012~\cite{moller2016} реализует коллективный подход к описанию структуры ядра при помощи продвинутого многопараметрического приближения капли несжимаемой жидкости, дополняя его феноменологическими микроскопическими поправками. Микроскопическая часть модели состоит из процедуры Струтинского~\cite{strutinsky1966, strutinsky1967} для учета оболочечных поправок и метода Липкина--Ногами~\cite{lipkin1960, nogami1964} для учета эффектов спаривания.

\subsubsection{Приближение жидкой капли}
Макроскопическая основа FRDM2012 развивает подход известной полуэмпирической формулы Бете-Вайцзеккера для вычисления энергии связи. Вместо самой энергии связи в FRDM2012 рассматривается величина потенциальной энергии атомного ядра, зависящая от числа протонов и нейтронов, а также формы нуклида. Выражение для вычисления потенциальной энергии строится в предположении, что ядро является каплей заряженной жидкости с некоторым дополнительными феноменологическими поправками, которые позволяют усредненно учесть некоторые микроскопические эффекты до применения более тонких методов, о которых будет говориться далее. В выражение входят вклады объемной и поверхностной энергии, энергии кулоновского взаимодействия c учетом поверхностных эффектов и поправкой на формфактор протона, энергия Вигнера, усредненная энергия спаривания. 

В FRDM2012 влияние деформаций обеспечено и коллективными, и микроскопическими вкладами в потенциальную энергию. Определение энергии основного состояния осуществляется путем минимизации потенциальной энергии по форме ядра, что позволяет определять не только ядерную массу, но и параметры деформации. В работе~\cite{moller2016} приводятся значения параметров $\beta_2$, $\beta_3$, $\beta_4$ и $\beta_6$, вычисленные при помощи модели FRDM2012.

\subsubsection{Метод оболочечных поправок Струтинского}
В подходе Струтинского оболочечные поправки рассматриваются как флуктуации над усредненным значением энергии системы, в котором отсутствуют вклады оболочечных эффектов. В предположении отсутствия остаточного протон-нейтронного взаимодействия эти поправки должны вычисляться отдельно для каждого типа нуклонов: $\delta E_n$ и $\delta E_p$ для нейтронов и протонов соответственно. Для обоих типов частиц оболочечная поправка определяется как разность суммы одночастичных энергий $\epsilon^q_i$ и усредненной энергии $\delta{E_q}$
\begin{equation}
\delta E_q(N_q, \beta) = \sum^{N_q}_{i = 1} \epsilon^q_i(\beta) - \tilde{E_q}(N_q, \beta),
\end{equation}
где под $\beta$ имеется в виду деформация, а под величиной $N_q$ --- число нуклонов соответствующего типа. 

Ключевой частью метода Струтинского является вычисление величины $\tilde{E_q}$, то есть усреднение суммы одночастичных энергий таким образом, чтобы в ней не осталось оболочечных эффектов. Вообще сумму одночастичных уровней можно выразить через плотность уровней $g_q$:
\begin{equation}
E_q = \int\displaylimits^\lambda_{-\infty} E g_q(E) dE,
\end{equation}
где $\lambda$ --- энергия Ферми, которую можно найти из условия сохранения числа частиц
\begin{equation}
N_q = \int\displaylimits^\lambda_{-\infty} g_q(E) dE
\end{equation}

Предполагается, что плотность уровней $g_q$ представляет собой сумму гладкой части $\tilde{g_q}$ и осциллирующей вокруг нее компоненты $\delta g_q$, соответствующей оболочечным эффектам: $g_q(E) = \tilde{g_q}(E) + \delta g_q(E)$. В таком случае вычисление усредненной энергии $\tilde{E_q}$ подразуемевает устранение осциллирующей части плотности уровней $\delta g_q$.

В методе усреднения Струтинского величина $\tilde{g_q}$ ищется в виде
\begin{equation}
\displaystyle
\tilde{g_q}(E) = \frac{1}{\gamma} 
\sum_i  f \left( \frac{E - \epsilon_i}{\gamma} \right),
\label{eq:smooth_density}
\end{equation}
где $\gamma$ --- параметр сглаживания, а $f(x)$ --- некая положительная аналитическая функция, нормированная на единицу, симметричная и имеющая максимум при $x = 0$. Если в оболочечной компоненте $\delta g_q(E)$ доминирует осцилляция с периодом $\hbar \omega$ (по сути соответствующая главным оболочкам), то, выбрав параметр $\gamma \gtrapprox \hbar \omega$, можно получить гладкую функцию $\tilde{g_q}(E)$.

Выбор усредняющей функции $f(x)$ ограничивается требованием на независимость величины $\tilde{g_q}(E)$ от величины $\gamma$. Может быть использовано следующее усреднение одночастичного спектра с гауссовой функцией:
\begin{equation}
\displaystyle
\tilde{g_q}(E) = \frac{1}{\gammaш\sqrt{\pi}} 
\sum_i P_M \left( \frac{E - \epsilon_i}{\gamma} \right) 
\exp{\left( \frac{E - \epsilon_i}{\gamma} \right)^2},
\end{equation}
где $P_M(x)$ --- полином степени $M$, который называют корректирующим полиномом. Его степень $M$ становится еще одним добавочным параметром после параметра сглаживания $\gamma$.

\subsubsection{Поправки для учета эффектов спаривания}
Используемый в FRDM2012 метод Липкина--Ногами~\cite{lipkin1960, nogami1964} основан на методе Бардина--Купера--Шриффера (БКШ)~\cite{bcs1957}, изначально разработанном для описания эффектов сверхпроводимости и применимого к вычислению парных корреляций в ядре. Волновые функции БКШ нарушают некоторые законы сохранения, и метод Липкина--Ногами решает эту проблему. Мы ограничимся качественным описанием подхода БКШ к описанию эффектов спаривания, так как даже в исходном виде он с успехом использовался для предсказания экспериментальных результатов.

Подход рассматривает протоны и нейтроны по отдельности, так как протон-нейтронные корреляции полагаются отсутствующими, а итоговая система уравнений одинакова для обоих типов нуклонов. Гамильтониан парного взаимодействия инвариантен относительно обращения времени, что означает двойное вырождение уровней, каждому из которых соответствуют два сопряженных состояния $\ket{k\sigma}$ ($\sigma = +,-$) между которыми и происходят парные взаимодействия. Полагается, что константа всех таких взаимодействий $G$ неизменна для всех переходов между состояниями. Скоррелированность нуклонов приводит к тому, что они приобретают дополнительную энергию и оказываются на более высоких уровнях. Существует возможность преодоления нуклонами энергии Ферми, из-за чего происходит размытие края заселенности энергетических уровней.

С помощью преобразования Боголюбова из операторов рождения и уничтожения нуклонов $a^+_{k\sigma}$ и $a_{k\sigma}$ строятся операторы квазичастиц, соответствующих парам спаренных нуклонов:
\begin{equation}
a_{k\sigma} = u_k b_{k -\sigma} + \sigma v_k b^+_{k\sigma},
\end{equation}
где $v_k$ соответствует доле заселенности $k$-го уровня квазичастицами и выполянется условие $u^2_k + v^2_k = 1$, а также закон сохранения числа нуклонов рассматриваемого типа:
\begin{equation}
n = 2\sum_k v_k^2
\end{equation}

Из этих положений о парных корреляциях в теории БКШ строится система уравнений:
\begin{equation}
\begin{gathered}
\displaystyle
n = \sum_k \left[ 1 - 
\frac{\varepsilon_k - \lambda}{\sqrt{(\varepsilon_k - \lambda)^2 + \Delta^2}} 
\right], \\
\frac{2}{G} = \sum_k \frac{1}{\sqrt{(\varepsilon_k - \lambda)^2 + \Delta^2}}, \\
v^2_k = \frac{1}{2} \left[ 1 - 
\frac{\varepsilon_k - \lambda}{\sqrt{(\varepsilon_k - \lambda)^2 + \Delta^2}} 
\right], \\
\varepsilon_k = e_k - G v_k^2,
\end{gathered}
\label{eq:bcs}
\end{equation}
где $e_k$ --- энергии одночастичных состояний, $\lambda$ --- химический потенциал, определяемый энергией верхнего занятого уровня, а в последние два уравнения записываются для всех состояний $k$. Корреляционная функция $\Delta$ определяется выражением 
\begin{equation}
\Delta = G \sum_k u_k v_k
\end{equation}
и в четных ядрах приблизительно соответствует половине энергии возбуждения первого состояния.

Система уравнений~(\ref{eq:bcs}) решается относительно $v_k$, $\lambda$ и $G$ (или $\Delta$, в зависимости от того, какая величина известна). Поправка, которую дают эффекты спаривания к результатам жидкокапельной модели, определяется выражением
\begin{equation}
\displaystyle
\delta E = \sum_k (2 v_k^2 - n_k)e_k - \frac{\Delta^2}{G} - G \sum_k v_k^4 +
\frac{1}{2} G \sum_k n_k,
\end{equation}
где $n_k$ --- число нуклонов на $k$-ом дважды-вырожденном уровне в отсутствие сил спаривания.



