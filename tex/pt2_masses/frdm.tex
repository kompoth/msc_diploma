\subsection{Макро-микроскопическая модель FRDM2012}
Массовая модель FRDM2012 реализует коллективный подход к описанию структуры ядра при помощи продвинутого многопараметрического приближения капли несжимаемой жидкости, дополняя его феноменологическими микроскопическими поправками. Микроскопическая часть модели состоит из процедуры Струтинского~\cite{strutinsky1966, strutinsky1967} для учета оболочечных поправок и метода Липкина-Ногами~\cite{lipkin1960, nogami1964} для учета эффектов спаривания.

\subsubsection{Приближение жидкой капли}

\subsubsection{Метод оболочечных поправок Струтинского}
В подходе Струтинского оболочечные поправки рассматриваются как флуктуации над усредненным значением энергии системы, в котором отсутствуют вклады оболочечных эффектов. В предположении отсутствия остаточного протон-нейтронного взаимодействия эти поправки должны вычисляться отдельно для каждого типа нуклонов: $\delta E_n$ и $\delta E_p$ для нейтронов и протонов соответственно. Для обоих типов частиц оболочечная поправка определяется как разность суммы одночастичных энергий $\epsilon^q_i$ и усредненной энергии $\delta{E_q}$
\begin{equation}
\delta E_q(N_q, \beta) = \sum^{N_q}_{i = 1} \epsilon^q_i(\beta) - \tilde{E_q}(N_q, \beta),
\end{equation}
где под $\beta$ имеется в виду деформация, а под величиной $N_q$ --- число нуклонов соответствующего типа. 

Ключевой частью метода Струтинского является вычисление величины $\tilde{E_q}$, то есть усреднение суммы одночастичных энергий таким образом, чтобы в ней не осталось оболочечных эффектов. Вообще сумму одночастичных уровней можно выразить через плотность уровней $g_q$:
\begin{equation}
E_q = \int\displaylimits^\lambda_{-\infty} E g_q(E) dE,
\end{equation}
где $\lambda$ --- энергия Ферми, которую можно найти из условия сохранения числа частиц
\begin{equation}
N_q = \int\displaylimits^\lambda_{-\infty} g_q(E) dE
\end{equation}

Предполагается, что плотность уровней $g_q$ представляет собой сумму гладкой части $\tilde{q}$ и осциллирующей вокруг нее компоненты $\delta g_q$, соответствующей оболочечным эффектам. В таком случае вычисление усредненной энергии $\tilde{E_q}$ подразуемевает устранение осциллирующей части плотности уровней. В методе усреднения Струтинского величина $\tilde{g_q}$ ищется в виде
\begin{equation}
\displaystyle
\tilde{g_q}(E) = \frac{1}{\gamma} 
\sum_i  f \left( \frac{E - \epsilon_i}{\gamma} \right),
\label{eq:smooth_density}
\end{equation}
где $\gamma$ --- параметр сглаживания, а $f(x)$ --- некая положительная аналитическая функция, нормированная на единицу, симметричная и имеющая максимум при $x = 0$. Если в оболочечной компоненте $\delta g_q(E)$ доминирует осцилляция с периодом $\hbar \omega$ (по сути соответствующая главным оболочкам), то, выбрав параметр $\gamma \gtrapprox \hbar \omega$, можно получить гладкую функцию $\tilde{g_q}(E)$.

Выбор усредняющей функции $f(x)$ ограничивается требованием на независимость величины $\tilde{g_q}(E)$ от величины $\gamma$. Может быть использовано следующее усреднение одночастичного спектра с гауссовой функцией:
\begin{equation}
\displaystyle
\tilde{g_q}(E) = \frac{1}{\gammaш\sqrt{\pi}} 
\sum_i P_M \left( \frac{E - \epsilon_i}{\gamma} \right) 
\exp{\left( \frac{E - \epsilon_i}{\gamma} \right)^2},
\end{equation}
где $P_M(x)$ --- полином степени $M$, который называют корректирующим полиномом. Его степень $M$ становится еще одним добавочным параметром после параметра сглаживания $\gamma$.

\subsubsection{Метод Липкина-Ногами}
