\subsection{Микроскопическая модель HFB-24}
Модель HFB-24~\cite{goriely2013} является очередным представителем семейства реализаций метода Хартри--Фока--Боголюбова с потенциалом взаимодействия Скирма, разрабатываемой группой авторов с 2003 года~\cite{samyn2003}. 

Метод Хартри--Фока позволяет приближенно решать уравнение Шредингера для системы многих тел, переходя к случаю движения невзаимодействующих частиц в самосогласованном потенциале. В методе Хартри--Фока--Боголюбова, развивающем этот подход, вместо волновых функций отдельных частиц рассматриваются волновые функции пар нуклонов, что позволяет учитывать в расчете эффект спаривания нуклонов. 

В настоящем разделе мы кратко опишем подход Хартри--Фока к описанию многочастичных квантовых систем и исторический вариант потенциала Скирма.

\subsubsection{Метод Хартри--Фока}
Изначально Д.~Хартри предложил описывать квантовую систему $A$ частиц (например, электронные оболочки атома или атомное ядро) с помощью факторизованной волновой функции $\psi(x_1, ..., x_A) = \phi_1(x_1) \cdot ... \cdot \phi_A(x_A)$, где $x_j$ --- пространственные координаты. Недостатком этой функции является отсутствие симметрии относительно перестановки частиц, неразличимых электронов и нуклонов. В дальнейшем В.А.~Фок предложил использовать детерминант Слэтера:
\begin{equation}
\displaystyle
\psi(\xi_1, ..., \xi_A) = \frac{1}{\sqrt{A!}} \cdot \text{det} 
\left[ \phi_i (\xi_j) \right],
\end{equation}
где $\xi_j$ обозначает пространственные, спиновые и изоспиновые координаты. Эта функция уже правильно учитывает неразличимость электронов в атомных оболочках или нуклонов в ядре.

Оператор Гамильтона многочастичной квантовой системы может быть записан в виде суммы оператора кинетической энергии и оператора взаимодействия частиц между собой. С учетом двух- и трехчастичных взаимодействий можно записать следующее выражение для полной энергии системы:
\begin{equation}
E_{HF} = \expval{T + V}{\psi} = \sum_i \expval{t_i}{i} 
  + \sum_{i < j} \expval{v_{ij}}{ij}
  + \sum_{i < j < k} \expval{v_{ijk}}{ijk}, 
\label{eq:hs-energy}
\end{equation}
где $T$ --- оператор кинетической энергии, а $V$ --- оператор взаимодействий между нуклонами.

С помощью вариационного принципа и условия нормировки волновых функций $\sum_i \int \phi_i^* \phi_i dx = A$ может быть составлено уравнение на волновые функции системы, при которых реализуется состояние с наименьшей энергией, то есть в случае с атомным ядром основное состояние:
\begin{equation}
\displaystyle
\frac{\delta}{\delta \phi^*_\alpha}
\left[ E_{HF} - \sum_i e_i \int |\phi_i|^2 dx \right] = 0,
\end{equation}
где множители Лагранжа $e_i$ окажутся энергиями отдельных частиц системы. Действительно, выполнив дифференцирование, можно получить систему уравнений уравнений, соответствующих уравнениям Шредингера для волновой функции каждой частицы:
\begin{equation}
h \ket{\phi_i} = e_i \ket{\phi_i}
\end{equation}

Одночастичные гамильтонианы $h_i$ вместо потенциалов взаимодействия частиц будут содержать лишь взаимодействие с самосогласованным полем, являющимся усреднением парных взаимодействий. Сложность заключается в том, что это поле в общем случае является функционалом одночастичных волновых функций, поэтому его и называют самосогласованным. Из-за этого система уравнений Хартри-Фока обычно решается методом последовательных приближений, начиная с некой приближенной волновой функции системы.

\subsubsection{Потенциал Скирма}
В качестве потенциала двухчастичного взаимодействия $v_{ij}$ из выражения~(\ref{eq:hs-energy}) наиболее широко применяется потенциал Скирма. Он имеет следующий вид:
\begin{equation}
\begin{aligned}
\displaystyle
v_{12} &= t_0 (1 + x_0 P_\sigma) \delta_{12} \\
       &+ \frac{1}{2} t_1 (1 + x_1 P_\sigma) 
       \left[\bm{k}^{\dag 2} \delta_{12} + \delta_{12} \bm{k}^2\right] \\
       &+ t_2 (1 + x_2 P_\sigma) \bm{k}^\dag \delta_{12} \bm{k} \\
       %&+ \frac{1}{6} t_3 (1 + x_3 P_\sigma) 
       %\rho^\gamma(\frac{\bm{r}_1 + \bm{r}_2}{2}) \delta_{12} \\
       &+ i W_0 \bm{k}^\dag \delta_{12} \times 
       (\bm{\sigma}_1 + \bm{\sigma}_2) \bm{k},
\label{eq:skyrme-2}
\end{aligned}
\end{equation}
где $\delta_{12} = \delta(\bm{r}_1 - \bm{r}_2)$, а оператор $\bm{k}$ отвечает импульсу относительного движения частиц
\begin{equation}
\displaystyle
\bm{k} = \frac{i}{2} (\nabla_1 - \nabla_2)
\end{equation}
Оператор $P_\sigma$ является оператором обмена спинами:
\begin{equation}
\displaystyle
P_\sigma = \frac{1}{2} (1 + \bm{\sigma}_1 \bm{\sigma}_2)
\end{equation}
Величины $t_k$, $x_k$ и $W_0$ являются параметрами модели.

Потенциал двухчастичного взаимодействия~(\ref{eq:skyrme-2}) может быть получен разложением в ряд Тейлора модельного гауссового потенциала с эффектом обмена спинами и изоспинами в предположении короткодействия ($\mu \ll k_F^{-1}$):
\begin{equation}
v(r) = e^{-(r/\mu)^2} (W + B P_\sigma - H P_\tau - M P_\sigma P_\tau)
\end{equation}

Трехчастичное взаимодействие приближенно можно представить в виде силы с нулевым радиусом вида
\begin{equation}
v_{123} = t_3 \delta_{12} \delta_{23}
\end{equation}
Можно показать, что в четно-четных ядрах такой потенциал может быть сведен к двухчастичному потенциалу, зависящему от плотности ядерной материи $\rho(\bm{r})$:
\begin{equation}
\displaystyle
\frac{1}{6} t_3 (1 + x_3 P_\sigma) 
\rho(\frac{\bm{r}_1 + \bm{r}_2}{2}) \delta_{12}
\end{equation}
Этот потенциал обобщают на случай любых ядер введением параметра $\gamma$
\begin{equation}
\displaystyle
\frac{1}{6} t_3 (1 + x_3 P_\sigma) 
\rho^\gamma(\frac{\bm{r}_1 + \bm{r}_2}{2}) \delta_{12},
\label{eq:skyrme-3}
\end{equation}

Потенциал двухчастичного взаимодействия~(\ref{eq:skyrme-2}) с добавкой~(\ref{eq:skyrme-3}) называется потенциалом Скирма~\cite{skyrme1958}. В классическом варианте он обладает небольшим числом параметров: $x_0$, $x_1$, $x_2$, $x_3$, $t_0$, $t_1$, $t_2$, $t_3$, $W_0$ и $\gamma$, значения которых могут быть подсчитаны с помощью экспериментальных данных. В реалистичных ядерных моделях потенциал Скирма используется вместе с членами, отвечающими за кулоновское взаимодействие, эффект спаривания нуклонов и оболочечные поправки. В современных вариантах потенциал входят продвинутые поправки, позволяющие, например, учитывать вклад тензорных сил. Впервые ядерная модель, основанная на методе Хартри--Фока с потенциалом Скирма, была представлена в работе~\cite{vautherin1972}. Сегодня подход Скирма--Хартри--Фока является одним из наиболее широкоиспользуемых методов микроскопического описания атомных ядер, существует большое число его вариантов и параметризаций.
