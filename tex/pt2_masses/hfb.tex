\subsection{Микроскопическая модель HFB-24}
Модель HFB-24~\cite{goriely2013} является очередным представителем семейства реализаций метода Хартри--Фока--Боголюбова с потенциалом взаимодействия Скирма, разрабатываемой группой авторов с 2003 года~\cite{samyn2003}. 

Метод Хартри--Фока позволяет приближенно решать уравнение Шредингера для системы многих тел, переходя к случаю движения невзаимодействующих частиц в самосогласованном потенциале. В методе Хартри--Фока--Боголюбова, развивающем этот подход, вместо волновых функций отдельных частиц рассматриваются волновые функции пар нуклонов, что позволяет учитывать в расчете эффект спаривания нуклонов. 

В настоящем разделе мы кратко опишем подход Хартри--Фока к описанию многочастичных квантовых систем и исторический вариант потенциала Скирма.

\subsubsection{Метод Хартри--Фока}
Изначально Д.~Хартри предложил описывать квантовую систему $A$ частиц (например, электронные оболочки атома или атомное ядро) с помощью факторизованной волновой функции $\psi(x_1, ..., x_A) = \phi_1(x_1) \cdot ... \cdot \phi_A(x_A)$, где $x_j$ --- пространственные координаты. Недостатком этой функции является отсутствие симметрии относительно перестановки частиц, неразличимых электронов и нуклонов. В дальнейшем В.А.~Фок предложил использовать детерминант Слэтера:
\begin{equation}
\displaystyle
\psi(\xi_1, ..., \xi_A) = \frac{1}{\sqrt{A!}} \cdot \text{det} 
\left[ \phi_i (\xi_j) \right],
\end{equation}
где $\xi_j$ обозначает пространственные, спиновые и изоспиновые координаты. Эта функция уже правильно учитывает неразличимость электронов в атомных оболочках или нуклонов в ядре.

Энергия многочастичной системы, учитывая двух- и трехчастичные взаимодействия:
\begin{equation}
E = \expval{T + V}{\psi} = \sum_i \expval{T}{i} 
  + \sum_{i < j} \expval{v_{ij}}{ij} 
  + \sum_{i < j < k} \expval{v_{ijk}}{ijk}, 
\end{equation}
где $T$ --- оператор кинетической энергии, а $v_{ij}$ и $v_{ijk}$ --- операторы двух- и трехчастичных взаимодействий. От уравнения Шредингера с таким гамильтонианом с помощью вариационного принципа приходят к системе уравнений Хартри-Фока, соответствующих одночастичным уравнениям Шредингера в поле самосогласованной силы:
\begin{equation}
\left[ - \nabla \cdot \frac{\hbar^2}{2m_q^*} \nabla + U_q
+ W_q \left[l \times \sigma\right]\right] \phi_i = e_i \phi_i,
\end{equation}
где выражения для эффективной массы $m^*_q$, центральная $U_q$ и спин-орбитальная $W_q$ части потенциала зависят от выбора потенциалов $v_{ij}$ и $v_{ijk}$. Величины $e_i$ соответствуют энергиям частиц. Уравнения строятся для протонов и для нейтронов ($q = n, p$) и решаются относительно волновых функций частиц системы $\phi_i$. Обычно это делается методом последовательных приближений, начиная с какой-то модельной функции.

\subsubsection{Потенциал Скирма}
