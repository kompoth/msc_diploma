\subsection{Метод локальных массовых соотношений LMR2021}
В отличие от рассмотренных выше ядерных моделей, подход локальных массовых соотношений не описывает структуру и физику ядра, а основывается на закономерностях, связывающих массы соседних изотопов. По сути это очевидные дифференциальные соотношения, которые можно применять рекурсивно, начиная с ядер с известными энергиями связи и уходя сколь угодно далеко в область экзотических изотопов. Метод локальных массовых соотношений предложен в~\cite{garvey1966} и актуален до сих пор благодаря высокой точности (отмеченной, например, в~\cite{bao2014}) при сравнительной простоте вычислений и реализации. 

В настоящей работе используется таблица теоретических масс ядер LMR2021 \textbf{ССЫЛКА НА ЛЕНУ}, рассчитанная при помощи соотношения, которое связывает массы четырех соседних изотопов, расположенных на NZ-диаграмме в виде квадрата два на два, через энергию остаточного протон-нейтронного взаимодействия $\Delta_{np}$. Впервые эта величина использовалась для оценки энергий связи ядер в~\cite{janecke1974}.

Энергия взаимодействия протона и нейтрона в ядре с $N$ нейтронов и $Z$ протонов равна
\begin{equation}\label{eq:np-interaction}
  \begin{aligned}
    \Delta_{np}(N,Z) &= S_{np}(N,Z) - [S_{p}(N-1,Z) + S_{n}(N,Z-1)] = \\
    &= B(N,Z) + B(N-1,Z-1) - B(N,Z-1) - B(N-1,Z),
  \end{aligned}
\end{equation}
где $S_{np}$ и $S_{p}$ -- энергии отделения пары $np$ и протона соответственно, а $B(N,Z)$ --- энергия связи ядра с $N$ нейтронов и $Z$ протонов.

