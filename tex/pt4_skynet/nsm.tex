\subsection{Нуклеосинтез при слиянии нейтронных звезд}
На сегодняшний день слияние двух сверхкомпактных астрофизических объектов, нейтронных звезд или нейтронной звезды и черной дыры звездной массы, принято считать основным сценарием протекания $r$-процесса~\cite{korobkin2012}. В пользу слияния нейтронных звезд как важного источника $r$-изотопов говорит, например, высокая нейтроноизбыточность вещества, выбрасываемого в пространство (усредненное отношение числа протонов к общему числу нуклонов $Y_e \equiv \expval{Z/A} \approx 0.1$~\cite{kajino2019}), а также сходство распространенностей тяжелых изотопов в симуляциях этого сценария нуклеосинтеза с распространенностями в Солнечной системе (например,~\cite{freiberghaus1999,goriely2011,korobkin2012}).

\subsubsection{}

\subsubsection{Параметры моделирования}
Опираясь на результаты модели слияния нейтронных звезд~\cite{korobkin2012} и статью авторов библиотеки SkyNet~\cite{lippuner2015} о моделировании $r$-процесса в этом сценарии, мы остановились на следующих начальных макроскопических параметрах симуляции:
\begin{equation}
\displaystyle
T = 6~\text{ГК},\quad Y_e = 0.1,\quad s = 10 \frac{k_B}{\text{барион}}
\end{equation}

Исходные состав и плотность вещества рассчитывается с помощью уравнений статистического баланса (см. раздел~\ref{sec:nse}). Согласно~\cite{goriely2011,korobkin2012,theilemann2017}, перед началом $r$-процесса расширяющийся выброс вещества сливающихся нейтронных звезд сильно разогрет и находится в состоянии статистического баланса. Напомним, что для равновесной системы, в которой выходы прямых и обратных реакций компенсируют друг друга, можно получить связь между концентрациями изотопов и термодинамическими параметрами. Библиотека SkyNet позволяет с его помощью рассчитать исходные концентрации изотопов и недостающие термодинамические величины, в данном случае плотность.

Если вещество выброса расширяется равномерно~\cite{korobkin2012,lippuner2015}, то его плотность меняется обратно пропорционально кубу времени эволюции системы. В SkyNet для подобных сценариев предлагается кусочно-определенная зависимость:
\begin{equation}
\displaystyle
\rho(t) = \begin{cases}
\rho_0 e^{-t/\tau}, \quad t \leq 3\tau\\
\rho_0 \left( \frac{3\tau}{et} \right)^3, \quad t \geq 3\tau
\end{cases}
\end{equation}
У этой функции нет проблем со значением в момент $t=0$ и, как отмечается в~\cite{lippuner2015}, она все еще соответствует астрофизическому сценарию выброса вещества при слиянии нейтронных звезд. При расчете нуклеосинтеза мы использовали величину характерного времени $\tau = 10$~мс. Как уже говорилось в разделе~\ref{sec:eos}, для моделирования эволюции астрофизической ядерной системы достаточно задать временной профиль одной термодинамической величины, а все остальные можно рассчитывать на каждом шаге интегрирования с помощью уравнения состояния.
