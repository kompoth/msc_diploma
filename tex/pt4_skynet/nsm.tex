\subsection{Параметры моделирования}
Помимо астрофизических скоростей реакций, важными параметрами моделирования нуклеосинтеза являются термодинамические характеристики среды. В частности, скорость ядерной реакции является функцией температуры (см. формулу~(\ref{eq:rate})). Необходимо задать не только начальные значения этих величин, но и законы их изменения во времени. Все эти данные приходится извлекать из астрофизических моделей.

\subsubsection{Реалистичная модель слияния нейтронных звезд}
Опираясь на результаты модели слияния нейтронных звезд~\cite{korobkin2012} и статью авторов библиотеки SkyNet~\cite{lippuner2015} о моделировании $r$-процесса в сопровождающих слияние выбросах, мы остановились на следующих начальных значения температуры, средней доли протонов и энтропии:
\begin{equation}
\displaystyle
T = 6~\text{ГК},\quad Y_e = 0.1,\quad s = 10 \frac{k_B}{\text{барион}}
\end{equation}

Исходные состав и плотность вещества рассчитывается с помощью уравнений статистического баланса (см. раздел~\ref{sec:nse}). Согласно~\cite{goriely2011,korobkin2012,theilemann2017}, перед началом $r$-процесса расширяющийся выброс вещества сливающихся нейтронных звезд сильно разогрет и находится в состоянии статистического баланса. Напомним, что для равновесной системы, в которой выходы прямых и обратных реакций компенсируют друг друга, можно получить связь между концентрациями изотопов и термодинамическими параметрами. Библиотека SkyNet позволяет с его помощью рассчитать исходные концентрации изотопов и недостающие термодинамические величины, в данном случае плотность.

Если вещество выброса расширяется равномерно~\cite{korobkin2012,lippuner2015}, то его плотность меняется обратно пропорционально кубу времени эволюции системы. В SkyNet для подобных сценариев предлагается кусочно-определенная зависимость:
\begin{equation}
\displaystyle
\rho(t) = \begin{cases}
\rho_0 e^{-t/\tau}, \quad t \leq 3\tau\\
\rho_0 \left( \frac{3\tau}{et} \right)^3, \quad t \geq 3\tau
\end{cases}
\end{equation}
У этой функции нет проблем со значением в момент $t=0$ и, как отмечается в~\cite{lippuner2015}, она все еще соответствует астрофизическому сценарию выброса вещества при слиянии нейтронных звезд. При расчете нуклеосинтеза мы использовали величину характерного времени $\tau = 10$~мс. Как уже говорилось в разделе~\ref{sec:eos}, для моделирования эволюции астрофизической ядерной системы достаточно задать временной профиль одной термодинамической величины, а все остальные можно рассчитывать на каждом шаге интегрирования с помощью уравнения состояния.

\subsubsection{Каноническая модель $r$-процесса}
Отметим, что помимо реалистичных моделей $r$-процесса, воспроизводящих конкретные астрофизические явления, существуют и универсальные модели, параметры которых подбираются из требования эффективного протекания $r$-процесса. Ряд таких моделей описан в обзоре~\cite{arnould2007}. Их преимуществами являются простот и независимсоть от астрофизического сценария, что позволяет проводить пробные расчеты $r$-процесса, проверять модель и делать предварительные выводы о чувствительности $r$-процесса к выбору параметров.

В предыдущих наших работах~\cite{my-vestnik2021,my-iran2022,my-pos2022}, а также в работе, выполненной вместе с авторами массовой модели LMR2021~\cite{vladimirova2022}, для исследования влияния теоретических ядерных масс на выходы $r$-изотопов мы использовали каноническую модель, описанную в~\cite{arnould2007}, с некоторыми отличиями. В канонической модели $r$-процесса температура вещества и содержание в нем нейтронов постоянны на всем времени симуляции и достаточны, чтобы поддерживать высокие скорости реакций $(n,\gamma)$ и $(\gamma,n)$, превышающие скорости $\beta^-$-распадов. Исходное вещество является чистым ${}^{56}\text{Fe}$, моделируя таким образом железное ядро массивной звезды, взрывы которых рассматриваются как важный источник $r$-изотопов. В канонической модели учитываются только основные реакции $r$-процесса, а также спонтанное деление тяжелых ядер. Наконец, предполагается, что при $Z \geq 26$ реакции $(n,\gamma)$ и $(\gamma,n)$ находятся в состоянии статистического баланса, что обеспечивает устойчивость пути $r$-процесса.

Наша модель отличалась от канонической тем, что концентрация нейтронов была не постоянной, хоть и очень высокой (1000 нейтронов на 1 исходное ядро железа), учитывались все астрофизические реакции, представленные в библиотеке REACLIB~\cite{reaclib2010}, а условие баланса $(n,\gamma)$ и $(\gamma,n)$ не задавалось. Таким образом мы получили динамическую модель, в которой облако нейтронов постепенно истощалось и учитывался вклад всех возможных реакций. При этом в определенный момент симуляции мы наблюдали равновесие прямых и обратных реакций нейтронного захвата, при котором путь $r$-процесса на некоторое время стабилизировался.

В настоящей работе мы решили использовать модель, описанную в предыдущем подразделе, для получения более реалистичных результатов.
