\section{Моделирование $r$-процесса при слиянии нейтронных звезд} \label{skynet}
Целью настоящей работы является исследование чувствительности реалистичной модели $r$-процесса к неопределенностям теоретических масс нейтроноизбыточных ядер. Для этого мы использовали три библиотеки астрофизических скоростей реакций, рассчитанные нами на основе трех различных таблиц ядерных масс, для расчета эволюции концентраций изотопов в $r$-процессе. Метод моделирования нуклеосинтеза описан в разделе~\ref{sec:nucleosynthesis} этой работы. Модель, использованная нами для расчета скоростей ядерных реакций с вариацией массовой модели, описана в разделе~\ref{sec:statmodel}. В разделе~\ref{sec:massmodels} рассмотрены три модели ядерных масс, которые мы использовали для расчета скоростей реакций, а в разделе~\ref{sec:reaclib} описывается процедура составления баз данных этих скоростей. 

В этом разделе представлена конечная стадия работы: симуляции $r$-процесса, протекающего в выбросах звездного вещества при слиянии двух нейтронных звезд, с использованием трех различных баз данных и анализ полученных результатов. 

\input tex/pt4_skynet/nsm.tex
\input tex/pt4_skynet/results.tex
